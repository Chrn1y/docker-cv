%%%%%%%%%%%%%%%%%
% This is an sample CV template created using altacv.cls
% (v1.3, 10 May 2020) written by LianTze Lim (liantze@gmail.com). Now compiles with pdfLaTeX, XeLaTeX and LuaLaTeX.
%
%% It may be distributed and/or modified under the
%% conditions of the LaTeX Project Public License, either version 1.3
%% of this license or (at your option) any later version.
%% The latest version of this license is in
%%    http://www.latex-project.org/lppl.txt
%% and version 1.3 or later is part of all distributions of LaTeX
%% version 2003/12/01 or later.
%%%%%%%%%%%%%%%%

%% If you are using \orcid or academicons
%% icons, make sure you have the academicons
%% option here, and compile with XeLaTeX
%% or LuaLaTeX.
% \documentclass[10pt,a4paper,academicons]{altacv}

%% Use the "normalphoto" option if you want a normal photo instead of cropped to a circle
% \documentclass[10pt,a4paper,normalphoto]{altacv}

\documentclass[10pt,a4paper,ragged2e,withhyper]{altacv}

%% AltaCV uses the fontawesome5 and academicons fonts
%% and packages.
%% See http://texdoc.net/pkg/fontawesome5 and http://texdoc.net/pkg/academicons for full list of symbols. You MUST compile with XeLaTeX or LuaLaTeX if you want to use academicons.

% Change the page layout if you need to
\geometry{left=1.25cm,right=1.25cm,top=1.5cm,bottom=1.5cm,columnsep=1.2cm}

% The paracol package lets you typeset columns of text in parallel
\usepackage{paracol}

% Change the font if you want to, depending on whether
% you're using pdflatex or xelatex/lualatex
\ifxetexorluatex
  % If using xelatex or lualatex:
  \setmainfont{Roboto Slab}
  \setsansfont{Lato}
  \renewcommand{\familydefault}{\sfdefault}
\else
  % If using pdflatex:
  \usepackage[rm]{roboto}
  \usepackage[defaultsans]{lato}
  % \usepackage{sourcesanspro}
  \renewcommand{\familydefault}{\sfdefault}
\fi

% Change the colours if you want to
\definecolor{SlateGrey}{HTML}{2E2E2E}
\definecolor{LightGrey}{HTML}{666666}
\definecolor{DarkPastelRed}{HTML}{450808}
\definecolor{PastelRed}{HTML}{8F0D0D}
\definecolor{GoldenEarth}{HTML}{E7D192}
\colorlet{name}{black}
\colorlet{tagline}{PastelRed}
\colorlet{heading}{DarkPastelRed}
\colorlet{headingrule}{GoldenEarth}
\colorlet{subheading}{PastelRed}
\colorlet{accent}{PastelRed}
\colorlet{emphasis}{SlateGrey}
\colorlet{body}{LightGrey}

% Change some fonts, if necessary
\renewcommand{\namefont}{\Huge\rmfamily\bfseries}
\renewcommand{\personalinfofont}{\footnotesize}
\renewcommand{\cvsectionfont}{\LARGE\rmfamily\bfseries}
\renewcommand{\cvsubsectionfont}{\large\bfseries}


% Change the bullets for itemize and rating marker
% for \cvskill if you want to
\renewcommand{\itemmarker}{{\small\textbullet}}
\renewcommand{\ratingmarker}{\faCircle}

%% sample.bib contains your publications

\begin{document}
\name{Artyom Sushkov}
\tagline{Student}
%% You can add multiple photos on the left or right
%\photoR{2.8cm}{Globe_High}
% \photoL{2.5cm}{Yacht_High,Suitcase_High}

\personalinfo{%
  % Not all of these are required!
  \email{artyom.sushkov@gmail.com}
  \phone{+7 (917) 550-75-55}
  %\mailaddress{Åddrésş, Street, 00000 Cóuntry}
  \github{Chrn1y}
  \location{Moscow, Russia}
  %\homepage{www.homepage.com}
  %\twitter{@twitterhandle}
  %\linkedin{your_id}
  %% You MUST add the academicons option to \documentclass, then compile with LuaLaTeX or XeLaTeX, if you want to use \orcid or other academicons commands.
  % \orcid{0000-0000-0000-0000}
  %% You can add your own arbtrary detail with
  %% \printinfo{symbol}{detail}[optional hyperlink prefix]
  % \printinfo{\faPaw}{Hey ho!}[https://example.com/]
  %% Or you can declare your own field with
  %% \NewInfoFiled{fieldname}{symbol}[optional hyperlink prefix] and use it:
  % \NewInfoField{gitlab}{\faGitlab}[https://gitlab.com/]
  % \gitlab{your_id}
}

\makecvheader
%% Depending on your tastes, you may want to make fonts of itemize environments slightly smaller
% \AtBeginEnvironment{itemize}{\small}

%% Set the left/right column width ratio to 6:4.
\columnratio{0.6}

% Start a 2-column paracol. Both the left and right columns will automatically
% break across pages if things get too long.
\begin{paracol}{2}
\cvsection{Experience}

\cvevent{Band 7 on IELTS exam}{}{February 26 2021}{Moscow, Russia}
\divider

\cvevent{Participated in Tulahack}{}{November 13-15 2020}{Moscow, Russia}
\begin{itemize}
\item Managed to take the second place in the main nomination and won in several secondary nominations as part of a team "fulltiltclub"
\end{itemize}

\divider

\cvevent{Scored 100 out of 100 points on Russian State Exam in computer science}{}{June 2019}{Moscow, Russia}
\divider

\cvevent{Organised a hackathon in HSE Lyceum}{}{April 4-6 2019}{Moscow, Russia}
\begin{itemize}
\item The hackathon was organised together with Microsoft Student Partners
\end{itemize}



\divider


\cvevent{Participated in Urbanhack}{}{September 28-30 2018}{Moscow, Russia}
\begin{itemize}
\item The hackathon was oganised for 9-11 grades
\end{itemize}

%\divider

\cvsection{PROJECTS}




\cvproject{HSE Telegram Bot}{Bot for managing chats of HSE community}
\begin{itemize}
\item Current course project
\end{itemize}
\divider

\cvproject{Ngrokbot}{Telegram bot for managing ngrok during hackathons}
\divider

\cvproject{MRZ codes scanner}{Android app for scanning MRZ codes of documents}
\smallskip
\begin{itemize}
\item Was made together with team "fulltiltclub" during Tulahack
\end{itemize}
\divider

\cvproject{Librorum}{Android app for book recommendation}
\smallskip
\begin{itemize}
\item Was made as a course project in HSE Lyceum
\end{itemize}
\switchcolumn

\cvsection{Education}
\smallskip
%\divider
\cvevent{HSE Bachelor’s Programme 'Applied Mathematics and Information Science'}
{}
{2019 - 2023}{Moscow, Russia}

\cvevent{HSE Lyceum}{}
{2018 - 2019} {Moscow, Russia}


\cvsection{SKILLS}


%\textcolor{emphasis}{C++}
\mycvskill{C++}
\mycvskill{C}
\mycvskill{Python}
\mycvskill{Flask}
\mycvskill{peewee}
\mycvskill{SQLAlchemy}
\mycvskill{Linux}
\mycvskill{Docker}
\mycvskill{git}
\mycvskill{ASM}
\mycvskill{pyTelegramBotAPI}

\cvsection{LANGUAGES}

\cvskill{Russian}{5}
\cvskill{English}{4}

\end{paracol}


\end{document}
